\section*{Exercise 2}

\subsection*{a)}

The function $y_1(x_1, \dots , x_5)$ is defined as shown in the homework text \cite{homework-w4}. There are six groups of terms: $1$ constant, $5$ linear terms, $10$ two-factor interaction terms, $10$ three-factor terms, $5$ four-factor terms and $1$ five-factor term.

\subsection*{b)}

A more general definition of $y_1(x_1, \dots, x_n)$ with $n$ parameters would have $n + 1$ groups of terms. The number of terms in each group is given by a binomial coefficient, defined as $(n \quad k) = n!/k!(n-k)!$. The value $n$ is the number of parameters ($n = 5$ in our case). $k$ is the "order" of the group of terms, i.e. the number of parameters in that term. $k$ is synonymous with the leftmost column of Table \ref{tab:funko_terms}. As an example, a function dependent on $20$ parameters would have up to $15,504$ five-factor interaction terms. The total number of terms is $2^n$.

\subsection*{c)}

The factorial design for $y_1$ was evaluated, producing $32$ output values. The input values are encoded according to whether they have their maximum ($+$) or minimum ($-$) value. The relevant maximum and minimum parameter values are given in the assignment text. \cite[p.4]{homework-w4}

\begin{table}[h!]
\centering
\begin{tabular}{cr|cr|cr|cr}
	$x_1, x_2, x_3, x_4, x_5$ & $y_1$ &$x_1, x_2, x_3, x_4, x_5$ & $y_1$ &$x_1, x_2, x_3, x_4, x_5$ & $y_1$ &$x_1, x_2, x_3, x_4, x_5$ & $y_1$ \\
	\hline
	$-,-,-,-,-$ & $-22.03$ & $-,-,-,-,+$ &  $35.97$ & $-,-,-,+,-$ & $-17.52$ & $-,-,-,+,+$ & $-41.34$ \\
	$-,-,+,-,-$ & $-15.45$ & $-,-,+,-,+$ & $-52.61$ & $-,-,+,+,-$ & $-25.84$ & $-,-,+,+,+$ & $-30.78$ \\
	$-,+,-,-,-$ & $-20.89$ & $-,+,-,-,+$ & $-28.58$ & $-,+,-,+,-$ &  $-28.9$ & $-,+,-,+,+$ & $-41.44$ \\
	$-,+,+,-,-$ &  $-27.8$ & $-,+,+,-,+$ & $-40.12$ & $-,+,+,+,-$ & $-25.36$ & $-,+,+,+,+$ & $-28.21$ \\
	$+,-,-,-,-$ &  $-2.68$ & $+,-,-,-,+$ &  $-31.2$ & $+,-,-,+,-$ &  $-7.79$ & $+,-,-,+,+$ & $-11.37$ \\
	$+,-,+,-,-$ &  $-4.03$ & $+,-,+,-,+$ &  $-1.91$ & $+,-,+,+,-$ &  $13.47$ & $+,-,+,+,+$ &  $17.83$ \\
	$+,+,-,-,-$ & $-11.33$ & $+,+,-,-,+$ & $-20.85$ & $+,+,-,+,-$ &  $-4.65$ & $+,+,-,+,+$ &  $-5.06$ \\
	$+,+,+,-,-$ &   $2.93$ & $+,+,+,-,+$ &   $5.67$ & $+,+,+,+,-$ &  $29.82$ & $+,+,+,+,+$ &  $41.35$ \\
\end{tabular}
\caption{Outputs of the function $y1$ for all 32 possible input combinations.}
\label{tab:outputs_y1}
\end{table}

\subsection*{d)}

Analysis of Variance (ANOVA) is a method for determining the strength of different factor interactions. Each set of factors $x_i, x_j, \dots$ has a corresponding interaction term $P_{ij\dots}$, which is defined as:

\begin{equation}
	S_{ij\dots} = \frac{1}{n_L} \cdot \frac{(S_{ij\dots}^+ - S_{ij\dots}^-)^2}{n_k} \quad\rightarrow\quad P_{ij\dots} = \frac{S_{ij\dots}}{S_T} \cdot 100\ \% \quad \text{where} \quad S_T = \sum_{\forall y} (y - \bar{y})^2
\end{equation}

where $S_{ij\dots}^+$ is the sum of all outputs $y(x_1, \dots, x_n)$ where the parameters $x_i, x_j, \dots$ all have their maximum values, and $S_{ij\dots}^-$ is similarly defined for minimum values. Table \ref{tab:interactions_y1} shows all of these interaction terms for $y_1$. We notice that the greatest contributing term is the linear $P_1$, which is likely a result of the great difference between $x_{1,min} = -3.5$ and $x_{1,max} = 2.5$. Indeed, we see that the higher order terms which contribute most to the total include the effect of $x_1$. We can conclude that out of the five parameters, $y_1$ appears to be most sensitive to $x_1$.

\begin{table}[h!]
	\centering
	\begin{tabular}{c|ccccc}
		Order &          1          &          2           &          3           &           4           &           5            \\
		\hline
		      & $P_{1} = 34.71\ \%$ & $P_{12} = 8.42\ \%$  & $P_{123} = 3.04\ \%$ & $P_{1234} = 0.64\ \%$ & $P_{12345} = 0.79\ \%$ \\
		      & $P_{2} = 0.01\ \%$  & $P_{13} = 14.25\ \%$ & $P_{124} = 2.61\ \%$ & $P_{1235} = 1.47\ \%$ &          $-$           \\
		      & $P_{3} = 2.75\ \%$  & $P_{14} = 11.76\ \%$ & $P_{125} = 2.03\ \%$ & $P_{1245} = 1.06\ \%$ &          $-$           \\
		      & $P_{4} = 0.93\ \%$  & $P_{15} = 6.22\ \%$  & $P_{134} = 3.73\ \%$ & $P_{1345} = 2.04\ \%$ &          $-$           \\
		      & $P_{5} = 0.82\ \%$  & $P_{23} = 0.62\ \%$  & $P_{135} = 4.54\ \%$ & $P_{2345} = 0.28\ \%$ &          $-$           \\
		      &         $-$         & $P_{24} = 0.19\ \%$  & $P_{145} = 3.25\ \%$ &          $-$          &          $-$           \\
		      &         $-$         & $P_{25} = 0.24\ \%$  & $P_{234} = 0.28\ \%$ &          $-$          &          $-$           \\
		      &         $-$         & $P_{34} = 1.72\ \%$  & $P_{235} = 0.16\ \%$ &          $-$          &          $-$           \\
		      &         $-$         & $P_{35} = 0.14\ \%$  & $P_{245} = 0.02\ \%$ &          $-$          &          $-$           \\
		      &         $-$         &  $P_{45} = 0.0\ \%$  & $P_{345} = 0.64\ \%$ &          $-$          &          $-$           \\
		      \hline
		Total &     $39.22\ \%$     &     $43.58\ \%$      &     $20.31\ \%$      &      $5.49\ \%$       &       $0.79\ \%$
	\end{tabular}
	\caption{Interaction percentages for all terms in $y_1$.}
	\label{tab:interactions_y1}
\end{table}