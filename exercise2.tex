\section*{Exercise 2}

\subsection*{a)}

The function $y_1(x_1, \dots , x_5)$ is defined as a series of terms which are the product of a coefficient $b$ and a number of parameters $x_i$. There are six groups of terms with a different number of parameters. Table \ref{tab:funko_terms} summarizes these six groups of terms. The total number of terms is 32. The coefficients $b$ and the limits for the parameters $x$ are given. \cite{homework-w4}

\begin{table}[h!]
\centering
\begin{tabular}{c|lcl}
	Order & Form & \# of Terms & Name \\
	\hline
	$0$ & $b_0$ & $1$ & Constant term \\
	$1$ & $b_i x_i$ & $5$ & Linear term \\
	$2$ & $b_{ij} x_i x_j$ & $10$ & Two-factor interaction term \\
	$3$ & $b_{ijk} x_i x_j x_k$ & $10$ & Three-factor interaction term \\
	$4$ & $b_{ijkl} x_i x_j x_k x_l$ & $5$ & Four-factor interaction term \\
	$5$ & $b_{ijklm} x_i x_j x_k x_l x_m$ & $1$ & Five-factor interaction term \\
\end{tabular}
\caption{Summary of the terms that define $y1$.}
\label{tab:funko_terms}
\end{table}

\subsection*{b)}

A more general definition of $y_1(x_1, \dots, x_n)$ with $n$ parameters would have $n + 1$ groups of terms. The number of terms in each group is given by a binomial coefficient, defined as $(n \quad k) = n!/k!(n-k)!$. The value $n$ is the number of parameters ($n = 5$ in our case). $k$ is the "order" of the group of terms, i.e. the number of parameters in that term. $k$ is synonymous with the leftmost column of Table \ref{tab:funko_terms}. As an example, a function dependent on $20$ parameters would have up to $15,504$ five-factor interaction terms. The total number of terms is $2^n$.

\subsection*{c)}

The factorial design for $y_1$ was evaluated, producing $32$ output values. The input values are encoded according to whether they have their maximum ($+$) or minimum ($-$) value. The relevant maximum and minimum parameter values are given in the assignment text. \cite[p.4]{homework-w4}

\begin{table}[h!]
\centering
\begin{tabular}{cr|cr|cr|cr}
	$x_1, x_2, x_3, x_4, x_5$ & $y_1$ &$x_1, x_2, x_3, x_4, x_5$ & $y_1$ &$x_1, x_2, x_3, x_4, x_5$ & $y_1$ &$x_1, x_2, x_3, x_4, x_5$ & $y_1$ \\
	\hline
	$-,-,-,-,-$ & $-22.03$ & $-,-,-,-,+$ & $35.97$ & $-,-,-,+,-$ & $-17.52$ & $-,-,-,+,+$ & $-41.34$\\
	$-,-,+,-,-$ & $-15.45$ & $-,-,+,-,+$ & $-52.61$ & $-,-,+,+,-$ & $-25.84$ & $-,-,+,+,+$ & $-30.78$\\
	$-,+,-,-,-$ & $-20.89$ & $-,+,-,-,+$ & $-28.58$ & $-,+,-,+,-$ & $-28.9$ & $-,+,-,+,+$ & $-41.44$\\
	$-,+,+,-,-$ & $-27.8$ & $-,+,+,-,+$ & $-40.12$ & $-,+,+,+,-$ & $-25.36$ & $-,+,+,+,+$ & $-28.21$\\
	$+,-,-,-,-$ & $-2.68$ & $+,-,-,-,+$ & $-31.2$ & $+,-,-,+,-$ & $-7.79$ & $+,-,-,+,+$ & $-11.37$\\
	$+,-,+,-,-$ & $-4.03$ & $+,-,+,-,+$ & $-1.91$ & $+,-,+,+,-$ & $13.47$ & $+,-,+,+,+$ & $17.83$\\
	$+,+,-,-,-$ & $-11.33$ & $+,+,-,-,+$ & $-20.85$ & $+,+,-,+,-$ & $-4.65$ & $+,+,-,+,+$ & $-5.06$\\
	$+,+,+,-,-$ & $2.93$ & $+,+,+,-,+$ & $5.67$ & $+,+,+,+,-$ & $29.82$ & $+,+,+,+,+$ & $41.35$\\
\end{tabular}
\caption{Outputs of the function $y1$ for all 32 possible input combinations.}
\label{tab:outputs_y1}
\end{table}

\subsection*{d)}

Analysis of Variance (ANOVA) is a method for determining the strength of different factor interactions. Each set of factors $x_i, x_j, \dots$ has a corresponding interaction term $S_{ij\dots}$, which is defined as:

\begin{equation}
	S_{ij\dots} = \frac{1}{n_L} \cdot \frac{(S_{ij\dots}^+ - S_{ij\dots}^-)^2}{n_k}
\end{equation}

where $S_{ij\dots}^+$ is the sum of all outputs $y(x_1, \dots, x_n)$ where the parameters $x_i, x_j, \dots$ all have their maximum values, and $S_{ij\dots}^-$ is similarly defined for minimum values. From these we can define a normalized interaction term $P_{ij\dots}$ as:

\begin{equation}
	P_{ij\dots} = S_{ij\dots} / S_T \quad \text{where} \quad S_T = \sum_{\forall y} (y - \bar{y})^2
\end{equation}